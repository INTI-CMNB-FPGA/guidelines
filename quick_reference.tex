\documentclass{article}
\usepackage[a4paper,margin=2cm,landscape]{geometry}
\usepackage{hyperref}
\usepackage{listings}
\usepackage{multicol}
\usepackage{dirtree}  % texlive-generic-extra

%\linespread{0.5}

% Black section* with white text
\usepackage[explicit]{titlesec}
\usepackage{xcolor}
\titleformat{name=\section,numberless}{\normalfont\Large\bfseries}{}{0em}{\colorbox{black}{
  \parbox{\dimexpr\textwidth/3-8\fboxsep}{\textcolor{white}{#1}}
}}
% Vertical lines between columns
\setlength{\columnseprule}{1pt}
\def\columnseprulecolor{\color{black}}

% Header and footer
\usepackage{fancyhdr}
\pagestyle{fancy}

% Fixed Header parts
\lhead{CMNB - Quick Reference} % No left header
\rhead{Last update: 2017-05-25}

% Fixed Footer parts
\lfoot{https://github.com/INTI-CMNB-FPGA/guidelines}
\newcommand{\licenseurl}  {http://creativecommons.org/licenses/by/4.0/}
\newcommand{\licensedesc} {Creative Commons Attribution 4.0 International License}
\rfoot{This work is licensed under a \href{\licenseurl}{\licensedesc}.}
\cfoot{} % No central footer
% -------------------------------------------------------------------------------------------------
\begin{document}
% -------------------------------------------------------------------------------------------------
\chead{FPGA Project Conventions}

\begin{multicols}{3}
%    \dirtree{%
%      .1 project\_name.
%      .2 core.
%      .3 tb.
%      .2 doc.
%      .2 info.
%      .2 FPGA.
%      .2 helpers.
%      .2 hardware.
%      .2 softdware.
%      .2 firmware.
%      .2 verification.
%    }
  \section*{Directory structure}
    Mandatory:
    \begin{itemize}
      \item \textit{core:} HDL code of the core
      \item \textit{core/tb:} HDL code of the testbench
      \item \textit{doc:} generated documentation, such as specifications, manuals, notes, block diagrams, etc
    \end{itemize}
    Common used:
    \begin{itemize}
      \item \textit{info:} downloaded documentation, such as standards, application notes, datasheets, example code, etc
      \item \textit{FPGA:} synthesis projects distributed in directories called as \textit{boardname\_corename}
      \item \textit{helpers:} developed scripts
    \end{itemize}
    Others:
    \begin{itemize}
      \item \textit{hardware:} schematics and PCBs
      \item \textit{software:} developed for the project
      \item \textit{firmware:} code for processors
      \item \textit{verification:} formal verification
    \end{itemize}
    And needed git submodules such as \textit{fpga\_lib} in the root directory
  \section*{Files}
    \begin{itemize}
      \item Add a README.md (Markdown) at least in main directories
      \item README.md in the root directory must explain the project
      \item README.md in core directory must explain main blocks and how to use quickly
      \item Add Makefiles, Bash scripts or at least explain in README.md how to run simulation and synthesis
      \item Makefile in root directory must provide a target to analyze, compile and simulate all the project
      \item Add AUTHORS in the root directory, with name, email and what they work on
    \end{itemize}
  \section*{Names}
    \begin{itemize}
      \item Coherent and descriptive
      \item Avoid spaces and especial symbols (be careful with downloaded files)
      \item Do not use dates and version numbers
    \end{itemize}
  \section*{Considerations}
    \begin{itemize}
      \item For complex IP cores with multiple layers use several directories inside \textit{core}
            (with its own \textit{tb})
      \item Make one Library per project
      \item Make one Package per layer
      \item Use separated Packages for test if needed
      \item Automate with Makefiles and/or Bash scripts
    \end{itemize}
  \section*{Control version}
    \begin{itemize}
      \item Use Git as control version system
      \item Logs format:
        \begin{lstlisting}[]
Mandatory short description

Optional long description if
clarification is needed. Here
you can use * as bullets to
separate concepts.
        \end{lstlisting}
      \item Avoid multiple versions, use tags instead
      \item Use branches to add new features
      \item Try to use text formats also for documentation (LaTeX, flat of LibreOffice)
      \item Avoid generated and binary files (exception: production bitstream, delivered
            documentation, manufactured gerbers)
      \item Try to avoid name changes (\textit{git mv}), but if needed be clear in the log file
            (do a commit only indicating old names and new names)
    \end{itemize}
\end{multicols}
% -------------------------------------------------------------------------------------------------
\newpage
\chead{HDL Conventions}

\begin{multicols}{3}
  \section*{General}
    \begin{itemize}
      \item Use VHDL 93 (default) or Verilog 2001 (if justified)
      \item English for all (code, names, comments, doc, etc)
      \item Make useful comments
      \item Use meaningful names
      \item Never use reserved words as name
      \item Separate words with underscores
      \item Use positive (active high) logic if feasible
    \end{itemize}
  \section*{Files and Units}
    \begin{itemize}
      \item File extensions:
        \begin{itemize}
          \item \textit{.vhdl} for VHDL
          \item \textit{.v} for Verilog
          \item \textit{.vh} for Verilog Header
          \item \textit{.sv} for System Verilog
        \end{itemize}
      \item Use \textit{example.vhdl} for a Entity called \textit{example} and its Architecture[s]
      \item Use \textit{example.v} for a Module called \textit{example}
      \item Use \textit{example\_pkg.vhdl} for a Package called \textit{example}
            [and its Package Body]
      \item Use the suffix \textit{\_tb} for testbenches (file and Entity/Module names)
      \item Use \textit{example\_cfg.vhdl} for VHDL configurations (not commonly used)
      \item Architecture names:
        \begin{itemize}
          \item \textit{Structural} (instantiations), \textit{RTL}, \textit{Behav} (simulation)
          \item \textit{Power}, \textit{Area}, \textit{Speed} (if optimized for)
          \item Vendor or family names (if optimized for)
        \end{itemize}
      \item Considerations:
        \begin{itemize}
          \item Only one Entity and all its Architecture[s] per file
          \item Only one Module per file
          \item Only one Package [and its Package Body] per file
        \end{itemize}
    \end{itemize}
  \section*{Case}
    \begin{itemize}
      \item Lowercase is the default
      \item Uppercase for constants, generics/params and enumerations
      \item Mixed case for entity/module, architecture, packages and libraries names
    \end{itemize}
  \section*{Indentation}
    \begin{itemize}
      \item 3 spaces is the default
      \item 4 spaces with \textit{for}
      \item Never use tabs
      \item Line length less than 100 characters when feasible
      \item Use a space between a language statement and its bracket
      \item No space between names and its bracket
      \item Blocks:
        \begin{lstlisting}[language=vhdl]
if (...) open_symbol
   something
close_symbol else open_symbol
   something
close_symbol
        \end{lstlisting}
    \end{itemize}
  \section*{Header}
    \begin{lstlisting}[language=vhdl]
--
-- [Long] name of the core
--
-- Author(s):
-- * Name
--
-- Copyright Notice
-- License Notice
--
-- Description:
-- Long functional description.
--
    \end{lstlisting}
  \section*{Libraries (VHDL)}
    \begin{itemize}
      \item Only include one Library or Package per line
      \item Avoid to use deprecated IEEE libraries (Synopsis, Cadence, Mentor)
      \item You normally will use:
        \begin{lstlisting}[language=vhdl]
library IEEE;
use IEEE.std_logic_1164.all;
use IEEE.numeric_std.all;
        \end{lstlisting}
    \end{itemize}
  \section*{Port declaration}
    \begin{itemize}
      \item One port per line
      \item Comment \textbf{each} port
      \item Use only std\_logic and std\_logic\_vector (un/signed when meaningful)
      \item Records could be use in submodules (separated inputs and outputs)
    \end{itemize}
  \section*{Suffixes}
    \begin{itemize}
      \item \textit{\_i} input port
      \item \textit{\_o} output port
      \item \textit{\_io} bidirectional port
      \item \textit{\_n\_i} active low input port
      \item \textit{\_n\_o} active low output port
      \item \textit{\_r} internal register
      \item \textit{\_t} defined type
    \end{itemize}
  \section*{Labels}
    \begin{itemize}
      \item Use \textit{p\_} for processes
      \item Use \textit{i\_} for instantiations
      \item Use \textit{g\_} for generates
      \item Put the label in the same line of its target
      \item Use the full \textit{end} form
    \end{itemize}
  \section*{Common abbreviations}
    \begin{itemize}
      \item \textit{ack:} acknowledge
      \item \textit{addr:} address
      \item \textit{clk:} clock
      \item \textit{cnt, count:} counter
      \item \textit{ctrl:} control
      \item \textit{ena:} enable
      \item \textit{idx:} index
      \item \textit{irq:} interrupt
      \item \textit{num:} number (of)
      \item \textit{ptr:} pointer
      \item \textit{re, rena:} read enable
      \item \textit{rd:} read
      \item \textit{rst:} reset
      \item \textit{stb:} strobe
      \item \textit{we, wena:} write enable
      \item \textit{wr:} write
    \end{itemize}
\end{multicols}

% FSM one process
% * enumeration/parameters
% Clock starvation
% Component, functions and procedures in package
% Signal declaration. No global, no shared
% Variables for intermediate values

% -------------------------------------------------------------------------------------------------
%\newpage
%\chead{HDL Conventions}

%\begin{multicols}{3}
%  \section*{Ranges}
%    \begin{itemize}
%      \item Use a range for integers and subtypes (synthesis)
%      \item Use downto for std\_logic\_vector and subtypes
%    \end{itemize}
%\end{multicols}

% Full sync

% -------------------------------------------------------------------------------------------------
\end{document}
